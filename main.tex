\documentclass{article}
\usepackage[utf8]{inputenc}
\usepackage{amsmath, amssymb}
\usepackage[margin=1.0in]{geometry}
\usepackage{graphicx}
\usepackage{mathrsfs}
\usepackage{hyperref}

\title{Note of ROS Tutorial}
\author{Lectures by Dr. Pilwon Hur}
\date{August 17-18, 2016}

\usepackage{natbib}
\usepackage{graphicx}
\usepackage{amsthm}
 \usepackage{framed,color}
\definecolor{shadecolor}{rgb}{0.9,0.9,0.9}
\newtheorem{prop}{Proposition}
\usepackage{courier}
\begin{document}

\maketitle
\section{Introduction to ROS}
Let's create a catkin workspace:
\begin{shaded}
\noindent\texttt{\noindent
	\$\quad mkdir -p ~/catkin\_ws/src \quad \% create a directory\\
	 \$\quad cd ~/catkin\_ws/src\\
	 \$\quad catkin\_init\_workspace}	 
\end{shaded}
\subsection{What is ROS?}
This tutorial introduces a software platform
called \textbf{Robot Operating System}, or \textbf{ROS}. The official description of ROS is as follows:
\begin{displayquote}
\textit{ROS is an open-source, meta-operating system for your robot. It provides the
services you would expect from an operating system, including hardware abstraction,
low-level device control, implementation of commonly-used functionality,
message-passing between processes, and package management. It
also provides tools and libraries for obtaining, building, writing, and running
code across multiple computers}
\end{displayquote}
In general, a robot system consists of following components:
\begin{displayquote}
\begin{center}
Hardware + OS + Apps
\end{center}
\end{displayquote}
For users who want to focus on software (Apps) development, taking care of all the different types of communication between different components can be challenging. ROS, a flexible framework and also a collection of tools, libraries, and conventions that aim to simplify the task of creating complex and robust robot behavior across a wide variety of robotic platforms, provides the need for various types of communication as a platform.
\subsubsection{Brief history}
ROS was originally developed in 2007 under the name switchyard by the Stanford Artificial Intelligence Laboratory. From 2008 until 2013, development was performed primarily at Willow Garage, a robotics research institute/incubator. In February 2013, ROS stewardship transitioned to the Open Source Robotics Foundation. Table 1 shows the list of ROS versions developed so far. ROS is officially supported on Ubuntu. The current suggested version combinations for ROS and Ubuntu are Indigo $+$ Ubuntu 14.04.5 LTS and Kinetic Kame $+$ Ubuntu 16.04.1 LTS. We use the former option as the version choice through this tutorial.
\begin{table}
\caption{ROS versions}
\begin{center}
\begin{tabular}{llr} 
\toprule
\cmidrule(r){1-2}
Released Date    & Version \\
\midrule
August 2, 2010      & \textbf{C} Turtle    \\
March 2, 2011          &    \textbf{D}iamondback     \\
\vdots       & \vdots    \\
July 22, 2014 & \textbf{I}ndigo      \\
May 23, 2015,& \textbf{J}ade      \\
May 23, 2016,& \textbf{K}inetic Kame      \\
\bottomrule
\end{tabular}
\end{center}
\end{table}


%Ken
%______________Brief intro, and also recommended version


\subsection{Be familiar with Ubuntu}
\subsubsection{Recommended version of Ubuntu and installation info}
\subsubsection{Recommended tools for programming in Ubuntu}
\subsubsection{Frequently used Linux commands}

%Yi-tsen
%______________Terminator, Sublime, and other packages added later

\subsection{ROS installation and related setup}





\subsection{Turtlesim}

%Imporant: bash file and added ROS alias("cm,cw,cs...etc")
%Imporant: ROS-rqt-graph for work flow


\section{Details of Programming in ROS}
\subsection{ROS system structure overview}
ROS is a set of conventions and tools for building complex robot control systems.
In a system built on ROS, the primary building block of the system is the “node”.
A bunch of nodes working together concurrently make up a complete ROS-based
robot control system.
A node is a Linux process that performs some logically separable task. Nodes
communicate by message passing using a well defined protocol. This is primarily
intended for isolation and concurrency, but it has the neat side effect of
providing programming language interoperability—a node written in COBOL will
work fine (and you'll never even know) as long as it does the message passing
correctly




%Imporant: structures and units
%Imporant: structures and units

\subsection{First program with ROS: Hello World!}


%Imporant: create a package and make, cmakne catin make

\subsection{Program demonstration with Kinect}

%Maybe just video link and brief discription for viz and rqt_graph




\bibliographystyle{plain}
%\bibliography{references}
\end{document}