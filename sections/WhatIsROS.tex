\subsection{What is ROS?}
This tutorial introduces a software platform
called \textbf{Robot Operating System}, or \textbf{ROS}. The official description of ROS is as follows:
\begin{displayquote}
\textit{ROS is an open-source, meta-operating system for your robot. It provides the
services you would expect from an operating system, including hardware abstraction,
low-level device control, implementation of commonly-used functionality,
message-passing between processes, and package management. It
also provides tools and libraries for obtaining, building, writing, and running
code across multiple computers}
\end{displayquote}
In general, a robot system consists of following components:
\begin{displayquote}
\begin{center}
Hardware + OS + Apps
\end{center}
\end{displayquote}
For users who want to focus on software (Apps) development, taking care of all the different types of communication between different components can be challenging. ROS, a flexible framework and also a collection of tools, libraries, and conventions that aim to simplify the task of creating complex and robust robot behavior across a wide variety of robotic platforms, provides the need for various types of communication as a platform.
\subsubsection{Brief history}
ROS was originally developed in 2007 under the name switchyard by the Stanford Artificial Intelligence Laboratory. From 2008 until 2013, development was performed primarily at Willow Garage, a robotics research institute/incubator. In February 2013, ROS stewardship transitioned to the Open Source Robotics Foundation. Table 1 shows the list of ROS versions developed so far. ROS is officially supported on Ubuntu. The current suggested version combinations for ROS and Ubuntu are Indigo $+$ Ubuntu 14.04.5 LTS and Kinetic Kame $+$ Ubuntu 16.04.1 LTS. We use the former option as the version choice through this tutorial.
\begin{table}
\caption{ROS versions}
\begin{center}
\begin{tabular}{llr} 
\toprule
\cmidrule(r){1-2}
Released Date    & Version \\
\midrule
August 2, 2010      & \textbf{C} Turtle    \\
March 2, 2011          &    \textbf{D}iamondback     \\
\vdots       & \vdots    \\
July 22, 2014 & \textbf{I}ndigo      \\
May 23, 2015,& \textbf{J}ade      \\
May 23, 2016,& \textbf{K}inetic Kame      \\
\bottomrule
\end{tabular}
\end{center}
\end{table}